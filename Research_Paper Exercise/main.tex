\documentclass{article}
\usepackage{graphicx}
\usepackage{pgf-pie}
\usepackage{amsmath}

\title{Research Paper}
\author{Masud Rana}
\date{27 January 2025}

\begin{document}

\maketitle
\author{Author Name \\ \textit{Masud Rana, University of Rajshahi, Bangladesh} \\ \texttt{masudranaorg71@gmail.com}}
\date{JANUARY 2024}

\section{Abstract}
Mathematics is a branch of science that deals with numbers and operations. In this paper, we have discussed the major issues and difficulties faced by the students in learning mathematics. Finally, we have suggested some solutions to overcome the troubles faced by the students. Some teaching methods have been suggested to make the learning process more attentive and effective. This study gives the reasons for the students having negative attitude towards mathematics.
$Keywords: Learning issues in mathematics, solutions, methodology, teacher’s motivation.$

\section{Introduction}
Mathematics plays a vital role in our daily life. Mathematics is surrounded in our lives in many ways like civic, professional, cultural, etc. Mathematics is often a challenging subject for the students to choose it as their degree. Now-a-days, there is a poor performance among the students to choose mathematics as their career. Researchers identified that the difficulties in maths start from elementary school and typically it continues till their masters. The issue is that the students face mathematics as difficult in their elementary school itself and the progress continues in secondary school, so that they find mathematics as difficult to choose it as their degree course. The problem arises due to lack of perceptual skills, language, reasoning, memory, etc. Despite, to provide quality education in mathematics, there are a greater number of issues in teaching and learning mathematics. The issues are related to classroom management, culture, lack of efficient teachers, lack of teaching supports, lack of references, lack of time management, inequity, etc. Some places do not have proper classroom facilities, seating arrangement and finally there is a lack of technology for teaching and learning mathematics.
\cite{ayob2017factors}
\cite{abc}

\section{Students Perception on Mathematics}
Students’ belief and attitude towards mathematics teaching and learning plays an important role in mathematics education. Now-a days, the number of students choosing Mathematics in higher education goes on decreasing. Some students hate maths because they think it is very dull. They don’t get excited about numbers, tables or formulas. They always think that maths is difficult to understand and solve. One of the main reasons for students not
 hoosing mathematics is their lack of understanding and having a low content-based knowledge. This makes them have a negative perception about mathematics. Students attitude towards mathematics can be influence by many factors like gender, motivation, emotion, self-confidence, etc.,
\cite{blazar2017teacher}

\section{Objective of the study}
This study is to create awareness
\begin{itemize}
    \item on the impact of ‘Mathematics’ on the society at large and Education in particular suggesting solutions
    \item to get rid of the mind block towards Mathematics
    \item on the usefulness, power and beauty of Mathematics
\end{itemize}
\cite{dimartino2011attitude}

\section{Importance and Relevance of the study}
\begin{itemize}
    \item  The fear and unwillingness to study Mathematics among the high school students is increasing in an alarming rate.
    \item This, if not addressed upon might result in a chaotic situation leading to a generation which will be ‘Math’ illiterates.
    \item This study would definitely enable the students to think and create awareness on the importance of Mathematics in everyday life.
    \item This also will bring out the need for studying Mathematics which enhances the logical thinking, leading to creativity and innovation.
\end{itemize}

\section{Research Methodology}
\begin{itemize}
    \item Primary Data is collected from the respondents within Coimbatore city with the help of a structured questionnaire using Convenience sampling.
    \item Relevant Statistical tools are used to analyse the data collected regarding the various levels of Fear for Mathematics, the reasons and their effect on the Society and especially Education. 
    \item Data on the impact of Mathematics in Higher Education are also collected and analysed which when implemented will lead to increased creativity and innovative research.
\end{itemize}
\cite{gallagher2005gender}

\section{Statistical Tools}
Statistics is a branch of Mathematics that deals with Data. Some of the tools of Statistics that are used extensively in research can be listed as Correlation, Regression, ChiSquare Distribution, Analysis of Variance etc. In these days of Big Data and Data analytics, researchers seek the help of Statistical Software like R-Programming, SPSS, SAS and others. The ease with which the data are classified, grouped and analysed with this software is
remarkable and they are indeed a boon to the researchers. The use of diagrammatic representation and the Chi Square analysis for testing the relationships are highlighted in this
paper.

\section{Analysis and Interpretation}
\subsection{Statistical Tools}
The following statistical tools are used for the study:
\begin{itemize}
    \item Percentage Analysis
    \item Diagram Analysis
    \item Chi-square Analysis
\end{itemize}


\subsection{Percentage Analysis}
Simple percentage analysis is used by the researcher for the analysis and interpretation of the collected data.
\begin{equation}
   Percentage Analysis =  \frac{Actual responder}{Total responder} * 100%
\end{equation}
\subsection{Diagram Analysis}
The percentage are expressed in diagrams, since the visualization will give a clear picture about the data and it is also very attractive.
\subsection{Chi-square Analysis}
Chi-square analysis is used to test whether the two characteristics are independent or not. In other words, the chi-square test is used to test whether one of the factors has significant influence over the other factor. It is expressed as
\begin{equation}
   \chi^2 = \sum \frac{(0_i - e_i)^2}{e_i}
\end{equation}
Where, \\
    $o_i = is the observed frequency$ \\
    $e_i = is the expected frequency $

\subsection{Diagram Analysis}
\subsubsection{Distribution of people based on Age}
\begin{tabular}{ |c|c|c| }
    \hline
    Age Group & No. of Respondents & Percentage \\
    \hline
    13-15 & 109 & 55.33 \\
    16-18 & 88 & 44.67 \\
    Grand Total & 197 & 100\\
    \hline 
\end{tabular}

% \begin{figure}
%     \begin{tikzpicture}
%     \pie[ 
%         color = {
%         blue!100,
%         yellow!100},
%         text = legend]
%      {55.33/ 13-15, 44.67/16-18}
    
%     \end{tikzpicture}
% \end{figure}

\subsubsection{Class of Study}
\begin{tabular}{ |c|c|c| }
    \hline
    Class & No. of Respondents & Percentage \\
    \hline
    11th std -12th std & 81 & 41.12 \\
    9th std - 10th std & 116 & 58.88 \\
    Grand Total & 197 & 100\\
    \hline    
\end{tabular}
% \begin{figure}
%     \begin{tikzpicture}
%     \pie[ 
%         color = {
%         blue!100,
%         yellow!100},]
%      {58.388/9th std - 10th std , 41.12/11th std -12th std}
%     \end{tikzpicture}
% \end{figure}

\subsubsection{Reason for choosing Mathematics}
\begin{tabular}{ |c|c|c| }
    \hline
    Reason & No. of Respondents & Percentage \\
    \hline
    Easy & 20 & 10.15 \\
    Interesting & 130 & 65.99 \\
    Useful for life & 47 & 23.86 \\
    Grand Total & 197 & 100\\
    \hline    
\end{tabular}
% \begin{figure}
%     \begin{tikzpicture}
%     \pie[ 
%         color = {
%         blue!100,
%         yellow!100,
%         green!100},
%         text = legend]
%      {10.15/Easy , 65.99/Interesting, 23.86/Useful for life}
%     \end{tikzpicture}
% \end{figure}


\section{Chi-Square Analysis}
Based on Chi-square analysis, the following findings are made:
\begin{enumerate}
    \item There is an association between choosing favourite subject and board of studying.
    \item There is no association between board of study and choosing Mathematics in 11th
    \item There is an association between years of studying Mathematics and not liking Mathematics.
    \item There is no association between board of study and not getting good marks in Mathematics.
    \item There is an association between years of studying Mathematics and suggestions for making Mathematics easy.
\end{enumerate}

\section{Conclusion}
Mathematics has an essential part in our life. Mathematics develops in critical and logical thinking. In recent days, the students started disliking Mathematics in their secondary level or higher secondary level. Some students feel that teaching and learning process are now-a-days getting worser and this is the main cause of less interest in Mathematics. To overcome this situation, Teachers need to be creative through a various combination of integrated technology where it creates a positive atmosphere for the students to learnStudents suggest that if Mathematics is taught by connecting it with real life, the subject will be very easy. The enrolment in Mathematics can be increased by creating some innovative ideas and new teaching technology. The issues can be resolved through adequate use of technology.

\section{References}
\bibliographystyle{plain}
\bibliography{references}

\end{document}
